\documentclass[a4paper,11pt]{article}

\usepackage[utf8]{inputenc}
\usepackage[T1]{fontenc}
\usepackage[francais]{babel}
\usepackage{amsmath,amssymb}
\usepackage{fullpage}
\usepackage{xspace}
\usepackage{verbatim}
\usepackage{graphicx}
\usepackage{listings}
\usepackage[usenames,dvipsnames]{color}
\usepackage{url}


\lstset{basicstyle=\small\tt,
  keywordstyle=\bfseries\color{Orchid},
  stringstyle=\it\color{Tan},
  commentstyle=\it\color{LimeGreen},
  showstringspaces=false}

\title{\textbf{TP-Analyse Fonctionelle}}
\author{Victor Bros \and Lavainne Remy}
\date{\oldstylenums{07}/\oldstylenums{11}/\oldstylenums{2018}}

%--------------------Indention
\setlength{\parindent}{1cm}

\begin{document}
%--------------------Title Page
\maketitle
\underline{Exercice 3}
\bigbreak
1) On pose $\displaystyle f= \sum_{n \in \mathbb{Z}} f_n \delta_{na} \,\, \in D(\mathbb{R})$ avec $f_n$ N-périodique.
On pose $\displaystyle d = \sum_{k=0}^{3} \frac{1}{4}\delta_{ka} \,\, \in \mathcal{E}'(\mathbb{R})$ et
$\displaystyle g = f \ast d = \sum_{n \in \mathbb{Z}} g_n \delta_{na}$.
\newline
On a $\displaystyle \forall \varphi \in D(\mathbb{R})$ 
\begin{equation}
    \begin{aligned}
        <g,\varphi> &= <f\ast d, \varphi> \\
                    &= <f_{t},<d_{u},\varphi(t+u)>>\\
                    &= <f, \displaystyle{\sum_{k=0}^3 \frac{1}{4} \varphi(ka+.)}>\\
                    &= \displaystyle{\sum_{n \in \mathbb{Z}} \frac{f_n}{4} \sum_{k=0}^{3} \varphi((k+n)a)}\\
                    &=\displaystyle{\sum_{k=0}^{3} \sum_{n \in \mathbb{Z}} \frac{f_n}{4} \varphi((k+n)a)}\\
                    &=\displaystyle{\sum_{k=0}^{3} \sum_{n \in \mathbb{Z}} \frac{f_{n-k}}{4} \varphi(na)}\\
                    &=\displaystyle{\sum_{n \in \mathbb{Z}} \varphi(na) \sum_{k=0}^{3} \frac{f_{n-k}}{4}}
    \end{aligned}
\end{equation}
On a donc $\displaystyle g_n = \sum_{k=0}^{3} \frac{f_{n-k}}{4}$ qui est donc N-périodique car $f_n$ est N-périodique.
\newline
$\forall \varphi \in D(\mathbb{R})$
\begin{equation}
    \begin{aligned}
        <g,\varphi> &= \displaystyle{\sum_{n \in \mathbb{Z}} \sum_{k=0}^{3} \frac{f_{n-k}}{4} \varphi(na)}\\
                    &= \displaystyle{\sum_{k=0}^{3} \sum_{n \in \mathbb{Z}} \frac{f_{n-k}}{4} \varphi(na)}\\
                    &= \displaystyle{\sum_{k=0}^{3} \sum_{n \in \mathbb{Z}} \frac{f_{n+N-k}}{4} \varphi((n+N)a)}\\
                    &= <g, \tau_{-Na}\varphi>\\
                    &= <\tau_{Na}g,\varphi>
    \end{aligned}
\end{equation}
On a donc que g est Na-périodique.

\end{document}
