\documentclass[a4paper,11pt]{article}

\usepackage[utf8]{inputenc}
\usepackage[T1]{fontenc}
\usepackage[francais]{babel}
\usepackage{amsmath,amssymb}
\usepackage{fullpage}
\usepackage{xspace}
\usepackage{verbatim}
\usepackage{graphicx}
\usepackage{listings}
\usepackage[usenames,dvipsnames]{color}
\usepackage{url}


\lstset{basicstyle=\small\tt,
  keywordstyle=\bfseries\color{Orchid},
  stringstyle=\it\color{Tan},
  commentstyle=\it\color{LimeGreen},
  showstringspaces=false}

\title{\textbf{TP-Analyse Fonctionelle}}
\author{Victor Bros \and Lavainne Remy}
\date{\oldstylenums{07}/\oldstylenums{11}/\oldstylenums{2018}}

%--------------------Indention
\setlength{\parindent}{1cm}

\begin{document}
%--------------------Title Page
\maketitle
\underline{Exercice 1}
\bigbreak
1) D'après l'exercice I de la feuille de TD 5,
\newline
comme $\widehat{f} \in L^{2}(\mathbb{R})$ (car $supp(\widehat{f}) \subset [-\lambda_{c},\lambda_{c}]$ support compact),
\newline
le théorème de Shannon s'applique sous la condition  \fbox{$ a < \frac{1}{2\lambda_{c}}$} pour obtenir une reconstruction parfaite du signal.

\bigbreak
2) pour $f_{0}=\frac{F_0}{F_e}$ et $N=15$, on a
\begin{equation}
    \begin{aligned}
        S_{N}(\lambda) &= \displaystyle{a\sum_{k=-N}^{N-1} f(na) e^{-2i\pi na\lambda}}\\
                       &= \displaystyle{a\sum_{k=-15}^{14} e^{2i\pi (f_0 -\frac{\lambda}{F_e}) n}} \\
                       &= \fbox{\displaystyle{\frac{1}{F_e} \frac{e^{-30i\pi (f_0 -\frac{\lambda}{F_e}) } - e^{30i\pi (f_0 -\frac{\lambda}{F_e}) }}{1-e^{2i\pi (f_0 -\frac{\lambda}{F_e})}}}}\\
    \end{aligned}
\end{equation}

\bigbreak
\underline{Exercice 3}
\bigbreak
1) On pose $\displaystyle f= \sum_{n \in \mathbb{Z}} f_n \delta_{na} \,\, \in D(\mathbb{R})$ avec $f_n$ N-périodique.
On pose $\displaystyle d = \sum_{k=0}^{3} \frac{1}{4}\delta_{ka} \,\, \in \mathcal{E}'(\mathbb{R})$ et
$\displaystyle g = f \ast d = \sum_{n \in \mathbb{Z}} g_n \delta_{na}$.
\newline
On a $\displaystyle \forall \varphi \in D(\mathbb{R})$
\begin{equation}
    \begin{aligned}
        <g,\varphi> &= <f\ast d, \varphi> \\
                    &= <f_{t},<d_{u},\varphi(t+u)>>\\
                    &= <f, \displaystyle{\sum_{k=0}^3 \frac{1}{4} \varphi(ka+.)}>\\
                    &= \displaystyle{\sum_{n \in \mathbb{Z}} \frac{f_n}{4} \sum_{k=0}^{3} \varphi((k+n)a)}\\
                    &=\displaystyle{\sum_{k=0}^{3} \sum_{n \in \mathbb{Z}} \frac{f_n}{4} \varphi((k+n)a)}\\
                    &=\displaystyle{\sum_{k=0}^{3} \sum_{n \in \mathbb{Z}} \frac{f_{n-k}}{4} \varphi(na)}\\
                    &=\displaystyle{\sum_{n \in \mathbb{Z}} \varphi(na) \sum_{k=0}^{3} \frac{f_{n-k}}{4}}
    \end{aligned}
\end{equation}
On a donc $\displaystyle g_n = \sum_{k=0}^{3} \frac{f_{n-k}}{4}$ qui est donc N-périodique car $f_n$ est N-périodique.
\newline
$\forall \varphi \in D(\mathbb{R})$
\begin{equation}
    \begin{aligned}
        <g,\varphi> &= \displaystyle{\sum_{n \in \mathbb{Z}} \sum_{k=0}^{3} \frac{f_{n-k}}{4} \varphi(na)}\\
                    &= \displaystyle{\sum_{k=0}^{3} \sum_{n \in \mathbb{Z}} \frac{f_{n-k}}{4} \varphi(na)}\\
                    &= \displaystyle{\sum_{k=0}^{3} \sum_{n \in \mathbb{Z}} \frac{f_{n+N-k}}{4} \varphi((n+N)a)}\\
                    &= <g, \tau_{-Na}\varphi>\\
                    &= <\tau_{Na}g,\varphi>
    \end{aligned}
\end{equation}
On a donc que g est Na-périodique.
\bigbreak
2) On sait que $g_n$ et $f_n$ sont N-périodique. On peut donc calculer g en calculant N valeurs consécutives de $g_n$.
On calcule les $g_n$ pour $ n \in \left[ 0:N-1 \right]$ grâce  la TFD :$\widehat{g_n} = \widehat{(f * d)_n}$
On a donc $\widehat{g_n} = \widehat{f_n}\widehat{d_n}$ en prenant la TFD inverse on peut obtenir $g_n$


4) On a $\widehat{g_n} = \widehat{f_n}\widehat{d_n}$ donc si $\widehat{d}$ ne s'annule sur aucune de ses composantes
on a : $\widehat{f_n} = \frac{\widehat{f_n}}{\widehat{d_n}}$

5)


\bigbreak
\underline{Exercice 4}
\bigbreak

1) Pour $H(\lambda) = \chi_{[-1/4,1/4]}(\lambda), \forall \lambda \in [-1/2,1/2[$,
\newline
Soit $n \in \mathbb{Z}$
\begin{equation}
    \begin{aligned}
        h_n &= \displaystyle{\int_{-1/2}^{1/2} H(\lstset{\lambda}) e^{2i \pi n \lambda} \, \mathrm{d} \lambda}\\
            &= \displaystyle{\int_{-1/4}^{1/4} H(\lstset{\lambda}) e^{2i \pi n \lambda} \, \mathrm{d} \lambda}\\
            &= \displaystyle{\Bigg[ \frac{e^{2i \pi n \lambda}}{2i \pi n} \Bigg] ^{1/4}_{-1/4}}\\
            &= \displaystyle{\frac{e^{i\pi n/2} -e^{-i\pi n/2}}{2i\pi n}}\\
            &= \fbox{\displaystyle{\frac{sinc(n\pi /2)} {2}}}
    \end{aligned}
\end{equation}
\bigbreak

2) Pour $N = 15$, on réalise le décalage d'indice suggéré pour obtenir une version réalisable du filtre:
\newline
\begin{equation}
    \begin{aligned}
        H_n^*(\lambda) &= H_n(\lambda) e^{-2i \pi \frac{N-1}{2}\lambda}\\
        H_n^*(\lambda) &= h_0 + \sum_{k=1}^{(N-1)/2} h_n e^{-2i\pi n \lambda} +\sum_{k=1}^{(N-1)/2} h_{-n} e^{2i\pi n \lambda}\\
    \end{aligned}
\end{equation}
Par parité du sinus cardinal $h_n=h_{-n}$,
\begin{equation}
    \begin{aligned}
    H_n^*(\lambda) &= \sum_{k=1}^{(N-1)/2} 2 h_n cos(2\pi n \lambda) \in \mathbb{R}\\
\end{aligned}
\end{equation}
i.e.
\begin{equation}
    \begin{aligned}
    arg(H_n(\lambda)) &= arg(e^{2i \pi \frac{N-1}{2}\lambda})\\
                      &= \fbox{2\pi \frac{N-1}{2}\lambda}\\
    |H_n(\lambda)| &= \sum_{k=1}^{(N-1)/2} 2 h_n cos(2\pi n \lambda)
    \end{aligned}
\end{equation}

\bigbreak
3) En reprenant le calcul de la question 1),
\begin{equation}
    \begin{aligned}
    \Tilde{h_n} &= \displaystyle{\int_{-1/2}^{1/2} H(\lstset{\lambda}) e^{(2n-1) i\pi \lambda} \, \mathrm{d} \lambda}\\
                &= \fbox{\displaystyle{\frac{sinc((2n-1)\pi /4)} {2}}}
    \end{aligned}
\end{equation}
on conserve les $N$ valeurs de $\Tilde{h_n}$ en construisant une version réalisable du filtre en décalant les valeurs:
\begin{equation}
    \begin{aligned}
        H_n^*(\lambda) &= H_n(\lambda) e^{-2i \pi (\frac{N}{2}-1)\lambda}\\
    \end{aligned}
\end{equation}
En reprenant le calcul de 2),
\begin{equation}
    \begin{aligned}
    H_n^*(\lambda) &= \sum_{k=1}^{(N-1)/2} 2 \Tilde{h_n} cos(2\pi n \lambda) \in \mathbb{R}\\
\end{aligned}
\end{equation}
i.e.
\begin{equation}
    \begin{aligned}
    arg(H_n(\lambda)) &= arg(e^{2i \pi (\frac{N}{2}-1)\lambda})\\
                      &= \fbox{2i \pi (\frac{N}{2}-1)\lambda}\\
    |H_n(\lambda)| &= \sum_{k=1}^{(N-1)/2} 2 \Tilde{h_n} cos(2\pi n \lambda)
    \end{aligned}
\end{equation}

\bigbreak

4) Ainsi, pour $h_n$ défini, on a montré par 2) et 3) en fonction de la parité de $P$, que par les décalages d'indice des $h_n$ et modification si $P$ est paire, il est possible de se ramener par parité de $h_n e^{-2i\pi n \lambda}$ à une somme réelle. Il ne reste donc comme phase de la transformée de Fourier que le décalage d'indice de la somme qui correspond à une fonction linéaire de $\lambda$. Le filtre ainsi défini est à phase linéaire.



\end{document}
